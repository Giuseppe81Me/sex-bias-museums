\documentclass[11pt]{letter}
\usepackage[a4paper,left=2.5cm, right=2.5cm, top=1cm, bottom=1cm]{geometry}
\usepackage[osf]{mathpazo}
\usepackage{url}
\signature{Natalie Cooper on behalf of all coauthors}
\address{Natural History Museum, London \\ Cromwell Road \\ London, SW7 5BD \\ natalie.cooper@nhm.ac.uk}
\longindentation=0pt
\begin{document}

\begin{letter}{}
\opening{Dear Editors,}

Please find attached our article, `Sex biases in bird and mammal natural history collections', for consideration at Proceedings of the Royal Society B.

We all know that STEM subjects have serious issues with gender equality (as well as many other kinds of equality of course!). In addition, sex biases in research are literally killing women: women are more likely to die from heart attacks because the classical heart attack symptoms are those experienced by men; women are 47\% more likely to be seriously injured in car crashes because car safety features are designed for men; and drugs tested on male mice may have serious side effects for women when they interact with female hormones. A consideration of the gender dimension of research should therefore be essential for any rigorous scientific study.

Natural history collections are vital tools in biodiversity science. Often collected over hundreds of years, these collections are used to answer an array of questions, many vital to the continued survival of the planet. This includes studies of historical anthropogenic change, taxonomy, systematics, biogeography, genomics, development, parasitology, toxicology, morphological evolution and many more. However, in the interests of using a large sample, the sex of specimens is often overlooked. If natural history collections have sex biases, this has serious implications for the outcomes of this research. This has broad significance across biological sciences because of the breadth of uses of museum specimen data. 

In our paper we investigate sex ratios in over two million bird and mammal specimen records from five large international museums. We found a slight bias towards males in birds (40\% females) and mammals (48\% females), but this varied among orders, with interesting skew in a number of groups. Contrary to expectations, the proportion of female specimens has not significantly changed in 130 years, suggesting that we are getting no better at sex balanced collecting.

Male bias decreased in species with ``showy'' male traits like colourful plumage and horns, but  sexual dimorphism in body size had little effect. Most worryingly, male bias was strongest in name-bearing types; only 27\% of bird and 39\% of mammal types were female. Our results imply that previous studies may be impacted by undetected male bias, and that vigilance is required when using specimen data, collecting new specimens, and designating types. 
 
We look forward to hearing from you,


\closing{Yours sincerely,}


\end{letter}
\end{document}

