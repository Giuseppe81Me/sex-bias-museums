% Table 1

\begin{longtable}{p{3cm} p{16cm}}

\caption{Prominent uses of natural history specimens and how research outcomes may be influenced by sex biases.}\\ 
  
  % Header
  \hline
  \textbf{Use} & \textbf{Might sex biases in birds and mammals affect research outcomes?}\\ 

  % Body of table
  \hline
  Taxonomy & \textbf{Yes}. Sexes often have external differences; if these are used in the taxonomy of the group (e.g. male plumage colouration in birds\cite{paxton2009utility}) then taxonomy may be harder in one sex than another. If sexes are very different, and there is strong sex bias in name-bearing types, this may result in taxonomic inflation, with males and females being given different names (although this is nowadays rare).\\ 

  Systematics & \textbf{Maybe}. For standard molecular phylogenies, commonly used genes do not differ substantially among sexes (i.e. not to the extent that they would form different branches). In phylogenomic studies, however, gene trees may vary across a genome if sex chromosomes are included in the sample\cite{reddy2017}. Morphological phylogenies are likely to be most affected, as morphological characters can vary extensively between males and females. This also has implications for Total Evidence phylogenies that use both morphological and molecular data.\\ 

  Biogeography & \textbf{No}. Locally sexes may be spatially segregated (e.g. bat roosting sites\cite{altringham}), and have different dispersal rates\cite{pusey1987sex}, but sexes (necessarily) do not differ in terms of large scale biogeography.\\ 

  Genomics & \textbf{Yes}. Mammals and birds have chromosomal sex determination; in mammals XY male and XX female, in birds ZZ male and ZW female\cite{stevens1997sex}. The X and Z chromosomes are larger and have more genes than W and Y, thus genome size differs among sexes. Many genes are also sex-linked, so genomes will differ among sexes.\\ 

  Comparative anatomy & \textbf{Yes}. Males and females have internal and external anatomical differences, thus sex biases will influence comparative anatomy studies.\\ 

  Development & \textbf{Maybe}. In most vertebrates, early developmental stages are almost identical in males and females, however later development and sexual maturation involve highly divergent growth to result in adult sex differences \cite{badyaev2002growing}. If research is focused on early development or juvenile life-history stages then sex biases are unlikely to pose a problem.\\

  Morphological variability & \textbf{Maybe}. Perceived wisdom is that males are more variable than females. However, many detailed morphometric studies do not find this (e.g.\cite{polly1998variability,biswas2019} and references within) in birds or mammals when a large sample is included.\\ 

  Parasitology & \textbf{Yes}. Males are commonly more susceptible to infection, have lower immune function, and higher parasite loads than females\cite{zuk2009sicker}. This is likely due to testosterone inhibiting the immune system\cite{Klein:2016aa}. However, this is not true for all species and all kinds of parasites, e.g. breeding female birds had more blood parasites than males\cite{mccurdy1998sex}. Differences in either direction may cause parasite load and diversity to be misrepresented where collections are sex biased.\\ 

  Stable isotope ecology & \textbf{Yes}. The demands of producing eggs, brooding, pregnancy, and lactation can alter stable isotope ratios \cite{fuller2004nitrogen}. Many species also have sex segregated diets, e.g. leopards\cite{voigt2018sex}, and foraging ranges, so stable isotope ratios may vary among sexes even in non-breeding individuals.\\ 

  Toxicology & \textbf{Yes}. As above, sexes may differ in foraging ecology, which has consequences for contaminant burden. Furthermore, females may be able to eliminate some contaminants via eggs (e.g. mercury\cite{robinson2012sex}), an option not available to males.\\ 
  Morphological evolution & \textbf{Yes}. There is extensive sexual dimorphism in many of the traits used in studies of morphological evolution, for example body size\cite{Uyeda15908}, thus tempo and mode of evolution may vary with sex.\\ 

  \hline
\label{table_consequences}
\end{longtable}

